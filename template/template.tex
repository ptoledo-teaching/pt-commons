\documentclass[11pt,letterpaper]{article}

\usepackage{pt-commons}

% Document metadata
\version{1.0}
\build{auto}
\title{PT-Commons Package}
\titlesub{Feature Demonstration Document}
\titlesubsub{\docversion\ - \today}
\watermark{DEMO}

% Class/Course information (optional)
\classcode{PKG-001}
\classsemester{2025-2}
\classname{Package Documentation}

% Institution information (optional)
\workgroup{Development Team}
\department{LaTeX Templates}
\university{PT Teaching}

% Logo (optional - comment out if file doesn't exist)
% \logo{path/to/logo.png}

% Authors
\addauthor{Pablo}{Toledo}{pablo.toledo@example.com}{Lead Developer, PT Teaching}
\addauthor{Jane}{Doe}{jane.doe@example.com}{Contributor}

\begin{document}

\section{Introduction}

This document demonstrates all the features available in the \textbf{pt-commons} package. The package provides a comprehensive set of utilities for creating professional LaTeX documents with consistent styling and formatting.

\subsection{Document Metadata Commands}

The following metadata commands are available:

\begin{itemize}
    \item \inlinecode{\textbackslash version\{1.0\}} - Document version
    \item \inlinecode{\textbackslash build\{auto\}} - Build number (auto-increments)
    \item \inlinecode{\textbackslash title\{...\}} - Main title
    \item \inlinecode{\textbackslash titlesub\{...\}} - Subtitle
    \item \inlinecode{\textbackslash titlesubsub\{...\}} - Sub-subtitle
    \item \inlinecode{\textbackslash watermark\{DEMO\}} - Watermark text
\end{itemize}

\subsection{Author Management}

Authors are added using \inlinecode{\textbackslash addauthor}:

\begin{verbatim}
\addauthor{FirstName}{LastName}{email}{affiliation}
\end{verbatim}

Current authors: \titleauthorsnames{, }

\section{Colors}

The package defines several predefined colors:

\begin{itemize}
    \item \textcolor{ptred}{ptred} - Main red color
    \item \textcolor{ptdarkred}{ptdarkred} - Dark red variant
    \item \textcolor{ptlightblue}{ptlightblue} - Light blue color
    \item \textcolor{ptblue}{ptblue} - Main blue color
    \item \textcolor{ptdarkblue}{ptdarkblue} - Dark blue variant
    \item \textcolor{ptgreen}{ptgreen} - Main green color
    \item \textcolor{ptdarkgreen}{ptdarkgreen} - Dark green variant
    \item \textcolor{ptyellow}{ptyellow} - Main yellow color
    \item \textcolor{ptdarkyellow}{ptdarkyellow} - Dark yellow variant
    \item \textcolor{ptgray}{ptgray} - Light gray color
    \item \textcolor{ptdarkgray}{ptdarkgray} - Dark gray color
\end{itemize}

\section{Typography}

\subsection{Inline Code}

Use \inlinecode{\textbackslash inlinecode\{text\}} for inline code: \inlinecode{import numpy as np}

\subsection{Text Formatting}

The package uses \textbf{Fira Sans} as the default font with proper scaling and microtype optimization for improved readability.

\section{Tables}

The package provides enhanced table commands using tabularray:

\begin{center}
    \begin{tblr}{
        colspec = {X[2]X[1]X[1]},
        row{1} = {ptblue!42, c, font=\bfseries},
        hlines,
        vlines
        }
        \tableheader Feature & Status    & Version \\
        Color Definitions    & Available & 1.0     \\
        Table Styling        & Available & 1.0     \\
        Code Highlighting    & Available & 1.0     \\
        File Trees           & Available & 1.0     \\
        Graphics             & Available & 1.0     \\
    \end{tblr}
\end{center}

\subsection{Table Commands}

Special table commands available:
\begin{itemize}
    \item \inlinecode{\textbackslash tableheader} - Header row styling
    \item \inlinecode{\textbackslash tablesubheader} - Subheader styling
    \item \inlinecode{\textbackslash tablecellleft} - Left-aligned cell
    \item \inlinecode{\textbackslash tablecellcenter} - Center-aligned cell
    \item \inlinecode{\textbackslash tablecellright} - Right-aligned cell
    \item \inlinecode{\textbackslash tablecellbold} - Bold cell content
    \item \inlinecode{\textbackslash tablecellrotated} - Rotated cell (90°)
\end{itemize}

\section{File Trees}

Display directory structures using the \texttt{dirtree} environment with custom icons:

\begin{filetree}
    \ptcaption{filetree}{Example Project Structure}
    \label{tree:example}
    \begin{ptdirtree}
        \dirtree{%
            .1 \treeiconfirst{project/}.
            .2 \treeicon{src/}.
            .3 \treeicon{main.py}.
            .3 \treeicon{utils.py}.
            .3 \treeicon{config.json}.
            .2 \treeicon{docs/}.
            .3 \treeicon{README.md}.
            .3 \treeicon{manual.pdf}.
            .2 \treeicon{tests/}.
            .3 \treeicon{testmain.py}.
            .2 \treeicon{requirements.txt}.
            .2 \treeicon{setup.py}.
        }
    \end{ptdirtree}
\end{filetree}

Icons are automatically assigned based on file extensions:
\begin{itemize}
    \item Archive files (.zip, .tar, .gz)
    \item Audio files (.mp3, .wav)
    \item Code files (.py, .js, .cpp, .java, etc.)
    \item Office files (.xls, .doc, .ppt)
    \item Images (.jpg, .png, .svg)
    \item PDF files (.pdf)
    \item Video files (.mp4, .avi)
\end{itemize}

\section{Code Environments}

\subsection{Inline Code}

Use \inlinecode{import numpy as np} for short code snippets.

\subsection{Print Code (Black \& White)}

For code that should be printed without syntax highlighting:

\begin{ptprintcode}{python}
    def hello_world():
    """A simple greeting function."""
    print("Hello, World!")
    return True

    if __name__ == "__main__":
    hello_world()
\end{ptprintcode}

\subsection{Minted Code (with highlighting)}

For syntax-highlighted code (requires minted package):

\begin{minted}{python}
import numpy as np
import matplotlib.pyplot as plt

def plot_sine_wave(freq=1.0, phase=0.0):
    """Plot a sine wave with given frequency and phase."""
    x = np.linspace(0, 2*np.pi, 1000)
    y = np.sin(freq * x + phase)
    
    plt.figure(figsize=(10, 4))
    plt.plot(x, y, 'b-', linewidth=2)
    plt.grid(True, alpha=0.3)
    plt.xlabel('x')
    plt.ylabel('sin(x)')
    plt.title(f'Sine Wave (freq={freq}, phase={phase})')
    plt.show()
\end{minted}

\section{Graphics and Plots}

\subsection{Color Definitions}

The package provides color utilities for pgfplots and tikz graphics. All predefined colors (ptred, ptblue, ptgreen, etc.) are available for use in graphics.

\subsection{Example Plot}

\begin{center}
    \begin{tikzpicture}
        \begin{axis}[
                width=0.8\linewidth,
                height=6cm,
                xlabel={$x$},
                ylabel={$f(x)$},
                grid=major,
                legend pos=north west,
            ]
            \addplot[ptblue, thick, domain=0:10, samples=100] {sin(deg(x))};
            \addplot[ptred, thick, domain=0:10, samples=100] {cos(deg(x))};
            \addplot[ptgreen, thick, domain=0:10, samples=100] {sin(deg(x))*cos(deg(x))};
            \legend{$\sin(x)$, $\cos(x)$, $\sin(x)\cos(x)$}
        \end{axis}
    \end{tikzpicture}
\end{center}

\section{Lists and Enumeration}

\subsection{Itemized Lists}

Bullet points are automatically styled:

\begin{itemize}
    \item First level item
          \begin{itemize}
              \item Second level item
                    \begin{itemize}
                        \item Third level item
                    \end{itemize}
          \end{itemize}
    \item Another first level item
\end{itemize}

\subsection{Enumerated Lists}

Numbered lists are also supported:

\begin{enumerate}
    \item First numbered item
          \begin{enumerate}
              \item Nested numbered item
              \item Another nested item
          \end{enumerate}
    \item Second numbered item
    \item Third numbered item
\end{enumerate}

\section{Utility Commands}

\subsection{Date Formatting}

\begin{itemize}
    \item Current date (YYYY/MM/DD format): \todayymd
    \item Document version: \docversion
\end{itemize}

\subsection{List of Contents}

The package provides smart list commands that only appear if content exists:

\begin{itemize}
    \item \inlinecode{\textbackslash ptlistoftables} - Only shows if tables exist
    \item \inlinecode{\textbackslash ptlistoffigures} - Only shows if figures exist
    \item \inlinecode{\textbackslash ptlistofcodes} - Only shows if code listings exist
\end{itemize}

\section{Multi-language Support}

The package supports multiple languages:

\begin{itemize}
    \item Spanish (default): \texttt{spanish}
    \item English: \texttt{english} option
    \item Portuguese: \texttt{portuguese} option
    \item French: \texttt{french} option
\end{itemize}

Example usage: \texttt{\textbackslash usepackage[english]\{pt-commons\}}

\section{Customization Options}

\subsection{Package Options}

\begin{itemize}
    \item \texttt{english} - Use English language
    \item \texttt{portuguese} - Use Portuguese language
    \item \texttt{french} - Use French language
    \item \texttt{nominted} - Disable minted package (use basic verbatim instead)
\end{itemize}

\subsection{Document Structure Commands}

For class/course documents:
\begin{itemize}
    \item \inlinecode{\textbackslash classcode\{...\}} - Course code
    \item \inlinecode{\textbackslash classsemester\{...\}} - Semester
    \item \inlinecode{\textbackslash classname\{...\}} - Course name
\end{itemize}

For institutional documents:
\begin{itemize}
    \item \inlinecode{\textbackslash workgroup\{...\}} - Working group
    \item \inlinecode{\textbackslash department\{...\}} - Department
    \item \inlinecode{\textbackslash school\{...\}} - School
    \item \inlinecode{\textbackslash university\{...\}} - University
\end{itemize}

For corporate documents:
\begin{itemize}
    \item \inlinecode{\textbackslash corporationname\{...\}} - Company name
    \item \inlinecode{\textbackslash corporationdepartment\{...\}} - Department
\end{itemize}

\section{Advanced Features}

\subsection{Build Counter}

The package automatically tracks build numbers when \inlinecode{\textbackslash build\{auto\}} is used. The counter is stored in a file named \texttt{jobname.buildcount} or \texttt{jobnameVERSION.buildcount} if a version is specified.

\subsection{Watermarks}

Watermarks are automatically sized and rotated at 45°. The watermark includes version and build information if specified.

\subsection{Hyperlinks}

All hyperlinks are automatically colored using the ptlightblue color scheme for both internal links and URLs.

\section{Conclusion}

The pt-commons package provides a comprehensive set of features for creating professional LaTeX documents. It includes:

\begin{itemize}
    \item Consistent color schemes and typography
    \item Enhanced table formatting with tabularray
    \item File tree visualization with automatic icons
    \item Code highlighting support (minted)
    \item Multi-language support
    \item Document metadata and versioning
    \item Author management
    \item Automatic watermarking
    \item Build counter tracking
    \item Customizable styling
\end{itemize}

For more information, visit the project repository or consult the package documentation.

\section*{Package Information}

\begin{itemize}
    \item \textbf{Package}: pt-commons
    \item \textbf{Version}: 1.1
    \item \textbf{Date}: 2025/04/21
    \item \textbf{Author}: PT Teaching
    \item \textbf{License}: Open Source
\end{itemize}

\end{document}